% Options for packages loaded elsewhere
\PassOptionsToPackage{unicode}{hyperref}
\PassOptionsToPackage{hyphens}{url}
%
\documentclass[
]{article}
\usepackage{lmodern}
\usepackage{amsmath}
\usepackage{ifxetex,ifluatex}
\ifnum 0\ifxetex 1\fi\ifluatex 1\fi=0 % if pdftex
  \usepackage[T1]{fontenc}
  \usepackage[utf8]{inputenc}
  \usepackage{textcomp} % provide euro and other symbols
  \usepackage{amssymb}
\else % if luatex or xetex
  \usepackage{unicode-math}
  \defaultfontfeatures{Scale=MatchLowercase}
  \defaultfontfeatures[\rmfamily]{Ligatures=TeX,Scale=1}
\fi
% Use upquote if available, for straight quotes in verbatim environments
\IfFileExists{upquote.sty}{\usepackage{upquote}}{}
\IfFileExists{microtype.sty}{% use microtype if available
  \usepackage[]{microtype}
  \UseMicrotypeSet[protrusion]{basicmath} % disable protrusion for tt fonts
}{}
\makeatletter
\@ifundefined{KOMAClassName}{% if non-KOMA class
  \IfFileExists{parskip.sty}{%
    \usepackage{parskip}
  }{% else
    \setlength{\parindent}{0pt}
    \setlength{\parskip}{6pt plus 2pt minus 1pt}}
}{% if KOMA class
  \KOMAoptions{parskip=half}}
\makeatother
\usepackage{xcolor}
\IfFileExists{xurl.sty}{\usepackage{xurl}}{} % add URL line breaks if available
\IfFileExists{bookmark.sty}{\usepackage{bookmark}}{\usepackage{hyperref}}
\hypersetup{
  pdftitle={Apresentação do curso},
  hidelinks,
  pdfcreator={LaTeX via pandoc}}
\urlstyle{same} % disable monospaced font for URLs
\usepackage[margin=1in]{geometry}
\usepackage{longtable,booktabs}
\usepackage{calc} % for calculating minipage widths
% Correct order of tables after \paragraph or \subparagraph
\usepackage{etoolbox}
\makeatletter
\patchcmd\longtable{\par}{\if@noskipsec\mbox{}\fi\par}{}{}
\makeatother
% Allow footnotes in longtable head/foot
\IfFileExists{footnotehyper.sty}{\usepackage{footnotehyper}}{\usepackage{footnote}}
\makesavenoteenv{longtable}
\usepackage{graphicx}
\makeatletter
\def\maxwidth{\ifdim\Gin@nat@width>\linewidth\linewidth\else\Gin@nat@width\fi}
\def\maxheight{\ifdim\Gin@nat@height>\textheight\textheight\else\Gin@nat@height\fi}
\makeatother
% Scale images if necessary, so that they will not overflow the page
% margins by default, and it is still possible to overwrite the defaults
% using explicit options in \includegraphics[width, height, ...]{}
\setkeys{Gin}{width=\maxwidth,height=\maxheight,keepaspectratio}
% Set default figure placement to htbp
\makeatletter
\def\fps@figure{htbp}
\makeatother
\setlength{\emergencystretch}{3em} % prevent overfull lines
\providecommand{\tightlist}{%
  \setlength{\itemsep}{0pt}\setlength{\parskip}{0pt}}
\setcounter{secnumdepth}{-\maxdimen} % remove section numbering
\ifluatex
  \usepackage{selnolig}  % disable illegal ligatures
\fi

\title{Apresentação do curso}
\usepackage{etoolbox}
\makeatletter
\providecommand{\subtitle}[1]{% add subtitle to \maketitle
  \apptocmd{\@title}{\par {\large #1 \par}}{}{}
}
\makeatother
\subtitle{STT0408 - Fundamentos de Engenharia de Transportes}
\author{true}
\date{}

\begin{document}
\maketitle

\hypertarget{programa-resumido}{%
\subsection{1. Programa Resumido}\label{programa-resumido}}

\begin{itemize}
\tightlist
\item
  Os sistemas de transporte: componentes, organização e importância
  econômica
\item
  Locomoção de veículos ferroviários
\item
  Locomoção de veículos rodoviários
\item
  Fluxo de veículos e seu controle
\item
  Capacidade viária
\item
  Fluxo de veículos em cruzamentos de vias
\end{itemize}

\hypertarget{avaliauxe7uxe3o-do-aprendizado}{%
\subsection{1.1 Avaliação do
Aprendizado}\label{avaliauxe7uxe3o-do-aprendizado}}

\begin{itemize}
\tightlist
\item
  Duas provas regulares
\item
  Listas de exercícios
\item
  Projeto de curso
\item
  Prova substitutiva: \emph{toda a matéria}
\end{itemize}

\hypertarget{projeto-de-curso}{%
\subsubsection{1.2 Projeto de Curso}\label{projeto-de-curso}}

\begin{itemize}
\tightlist
\item
  Aplicação na prática dos conceitos aprendidos
\item
  Até 2 pessoas
\item
  \emph{Relatórios parciais} em datas preestabelecidas
\item
  \emph{Relatório final} com revisão dos relatórios parciais
\end{itemize}

\[NF_{proj} = \frac{\sum NP_{relatorio}}{n}\]

\hypertarget{cuxe1lculo-da-muxe9dia-final}{%
\subsubsection{1.3 Cálculo da Média
Final}\label{cuxe1lculo-da-muxe9dia-final}}

A média final é calculada pela seguinte equação e os coeficientes são
dados a seguir.

\[ MF = \alpha \cdot \beta \cdot (0,5 \cdot MP + 0,5 \cdot NF_{proj}) \]

\begin{longtable}[]{@{}ll@{}}
\toprule
Condição & \(\alpha\)\tabularnewline
\midrule
\endhead
\(NF_{proj} < 5,0\) & 0,0\tabularnewline
\(MP < 3,0\) & 0,0\tabularnewline
\(N.Provas < 2\) & 0,0\tabularnewline
\(NF_{proj} \geq 5,0\) & 1,0\tabularnewline
\bottomrule
\end{longtable}

\begin{longtable}[]{@{}ll@{}}
\toprule
Listas entregues (N) & \(\beta\)\tabularnewline
\midrule
\endhead
\(0\% \leq NL \lt 20\%\) & 1,00\tabularnewline
\(20\% \leq NL \lt 40\%\) & 1,01\tabularnewline
\(40\% \leq NL \lt 60\%\) & 1,02\tabularnewline
\(60\% \leq NL \lt 80\%\) & 1,03\tabularnewline
\(80\% \leq NL \lt 100\%\) & 1,04\tabularnewline
\(NL = 100\%\) & 1,05\tabularnewline
\bottomrule
\end{longtable}

\hypertarget{datas-importantes}{%
\subsection{2. Datas Importantes}\label{datas-importantes}}

\begin{longtable}[]{@{}l@{}}
\toprule
Provas\tabularnewline
\midrule
\endhead
30 abril\tabularnewline
18 junho\tabularnewline
03 julho {[}subst{]}\tabularnewline
\bottomrule
\end{longtable}

\hypertarget{monitores-pae}{%
\subsection{3. Monitores PAE}\label{monitores-pae}}

\begin{itemize}
\tightlist
\item
  Andre Morelli
  \textless{}\href{mailto:andre.morelli@usp.br}{\nolinkurl{andre.morelli@usp.br}}\textgreater{}
\item
  Elaine Ribeiro
  \textless{}\href{mailto:elainerribeiro@usp.br}{\nolinkurl{elainerribeiro@usp.br}}\textgreater{}
\item
  Leandro Marcomini
  \textless{}\href{mailto:leandro.marcomini@usp.br}{\nolinkurl{leandro.marcomini@usp.br}}\textgreater{}
\item
  Paola Yumi
  \textless{}\href{mailto:paolayumi@usp.br}{\nolinkurl{paolayumi@usp.br}}\textgreater{}
\end{itemize}

\end{document}
